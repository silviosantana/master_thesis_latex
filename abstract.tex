\chapter{Abstract}
“Property-Driven Design (PDD)” is a top-down methodology for Register Transfer Level (RTL) design. It starts from an abstract description of the system, and it incorporates formal property checking methods early into the design process in order to establish a formal relationship between abstract model and RTL implementation. The proposed approach extends the PDD methodology so it can be efficiently employed in design of pipelined processors. In PDD, properties are automatically generated aiming to avoid high efforts associated with verification by property checking. This work presents an algorithm to convert these generated properties into a property set that captures the pipeline behaviour of a processor core.

At the end of the processor design flow, a formally verified RTL implementation is available along with a formal relationship between concrete implementation and abstract model. This allows the abstract model to be used beyond prototyping purposes, as it also represents a trusted reference of the design. The experiments present a case study for an existent RISC-V processor core implementation. The resulting property set generated from an abstract implementation of the core hold for the given RTL model establishing a formal relationship between the two models. The results also evidence the simulation \textit{speedup} of the abstract model in comparison to the RTL simulation, and a reduction on the complexity and effort for generating a property set for the pipeline processor in comparison to property verification on a bottom-up approach.