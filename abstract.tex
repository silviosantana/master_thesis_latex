\chapter{Abstract}

This work presents an algorithm to generate properties for pipelined processors. The proposed algorithm extends \TLREP{a Property-First design}{Property-Driven development, a design} flow where \TLINS{formal} properties are generated by a tool for a given system-level description of the design. These generated properties are converted into a pipeline property set which reflects the particularities of a pipeline processor behaviour. The property set is refined simultaneously with the RTL implementation on a \textit{refine-and-implement} process. In the end, when the complete set of refined properties hold to the implemented design, a trusted ESL model of the processor is obtained. \TLSAY{the ESL can be trusted} 

A case study for an existent RISC-V processor implementation is presented. A SystemC description of the processor core is implemented, and the algorithm is applied to create the set of pipeline properties. These properties are refined until they hold to the given RTL implementation. The experimental results evidence an improvement \TLSAY{Just reduced effort?} on the effort to create and refine the property set compared to a set of properties generated from the RTL implementation in a \textit{bottom-up} approach, and they confirm the simulation \textit{speedup} for the ESL model compared to the RTL simulation.

\TLSAY{It's an okay abstract but not a really good one. It's not general enough and it does not capture the ideas. For someone who does not know PDD it won't provide anything. Try to be more abstract (thats the hardest party of writing) ... usually it helps to only answer the questions "Why? Why is this important? Why is this interesting? Why is the problem for your solution interesting to look at} 