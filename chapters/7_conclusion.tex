% ----------------------------------------------------------
% Conclusion
% ----------------------------------------------------------
\chapter{Conclusion}
\label{chapter:conclusion}

In this work, an algorithm for property generation for pipelined processors was presented. This algorithm extends a Property-First design approach, called Property-Driven Design (PDD) methodology, to improve its applicability for pipelined processors. The methodology is based in a property checking formal verification method. For a given SystemC-PPA description of a processor, a set of properties is generated, and the proposed Pipeline Algorithm is used to create a set of pipeline properties to be later refined. The refinement of the property set is done simultaneously with the implementation of the RTL processor. At the end of the design flow, when a complete set of properties hold for the implemented RTL, an ESL model that is sound to the RTL implementation is obtained. Having a trusted system-level model allows its usage as a reference model the same manner as the RTL. \TLSAY{reads a little bit like your abstract. Maybe change the first couple of sentences ... also comment for you abstract apply here}

A case study for a RISC-V processor implementation, the RI5CY core, was performed in this work. A SystemC-PPA description for the base instruction set of the core was implemented, and a set of properties generated from it using the \textit{DeSCAM} tool. The Pipeline Algorithm was employed to merge the properties into a set of pipeline properties. For completeness, a set of \SSQED{} properties was written for the core as well. This property set was refined according to the signals and control registers of the given RTL processor implementation until all the properties hold. The experimental results show a \textit{speedup} of approximately 144~times from the simulation at ESL to the RTL simulation. The estimated ESL design effort was only 2~person~weeks, and the effort estimated for applying the Pipeline Algorithm and completely refine the property set was 8~person~days. For means of comparison a set of properties was created directly from the RTL implementation in a \textit{bottom-up} approach, and the estimated effort for creating property set was 1~person~month.

For future work, the proposed algorithm can be implemented and included into the DeSCAM tool so the process for generating properties for pipelined processors can be completely automatized. In the same manner, the generation of the \SSQED{} property set can be also implemented so a complete set of properties can automatically generated and then refined.